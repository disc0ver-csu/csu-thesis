%!TEX root = ../../csuthesis_main.tex

\chapter{引用文献标注}

文献标注和索引的处理一直是学术写作中一个繁琐步骤,特别是在word环境下。而在\LaTeX 中我们只需要编辑(或直接获取) \BibLaTeX 格式索引文件然后在正文中使用\cs{cite} \cs{citet}等指令进行引用标注就可以。注意,引文参考文献的每条都应该在正文中标注出来\cite{knuth1984texbook,__2011,__2014,__2015,__2016,__2017,_lpr_2019,_lpr_2020,_lpr_2020-1,_lpr_2020-2,_lpr_2021,balakrishnan_corfu_2012,balakrishnan_tango_2013,balakrishnan_virtual_2020,cidon_copysets_2013,csu__2020,ding_scalog_2020,hartman_zebra_1993,jia_boki_2021,kalia_design_2016,knuth1984texbook,lamport1994latex,lockerman_fuzzylog_2018,mehdi_i_2017,Pawlak1982Rough,pritchard1969statistical,wei_vcorfu_2017},下面介绍在文章中标注引用指令的具体使用方法。

参考文献格式应符合国标 GB/T 7714 的要求,模板使用 \BibLaTeX 配合 \pkg{biblatex-gbt7714-2015} 样式包,\footnote{\url{https://www.ctan.org/pkg/biblatex-gb7714-2015}}控制参考文献的输出样式,由于该样式包 2016年后加入CTAN,如果你本地的\TeX 环境较旧,建议参考\url{https://github.com/disc0ver-csu/csu-thesis} 环境配置部分安装较新的\TeX 环境。

\section{顺序编码}

根据学校要求,参考文献标注用中括号上标形式进行标注。使用方式与效果如下表所展示

\begin{tabular}{l@{\quad$\Rightarrow$\quad}l}
    \verb|\cite{knuth1984texbook}|               & \cite{knuth1984texbook}               \\
    \verb|\citet{knuth1984texbook}|              & \citet{knuth1984texbook}              \\
    \verb|\citep{knuth1984texbook}|              & \citep{knuth1984texbook}              \\
    % 暂不支持
    % \verb|\cite[42]{knuth1984texbook}|           & \cite[42]{knuth1984texbook}           \\
    \verb|\cite{knuth1984texbook,lamport1994latex}| & \cite{knuth1984texbook,lamport1994latex} \\
\end{tabular}

\section{获取BibTeX格式索引}

获取参考文献的 BibTeX 格式索引有两种方式:

\begin{itemize}
    \item 通过Google Scholare或者百度学术等学术文献搜索引擎获取,自行编辑 .bib 文件
    \item 通过Zotero等学术文献整理软件,添加所有的引用文献至库中,导出对应的 .bib 文件
\end{itemize}

编译带参考文献的文章时,我们需要两次编译过程。我们提供了对应的自动化脚本,以及配合vscode latex插件的任务流程,
帮助模板使用者进行编译。

\section{参考文献插入示例}

LaTeX\cite{lamport1994latex}插入参考文献最方便的方式是使用bibliography\cite{pritchard1969statistical},大多数出版商的论文页面\cite{lamport1994latex,pritchard1969statistical}都会有导出bib格式参考文献的链接,把每个文献的bib放入``csuthesis\_main.bib'',然后用bibkey即可插入参考文献。

中南大学由原湖南医科大学、长沙铁道学院与中南工业大学于2000年4月合并组建而成。原中南工业大学的前身为创建于1952年的中南矿冶学院,原长沙铁道学院的前身为创建于1953年的中南土木建筑学院,两校的主体学科最早溯源于1903年创办的湖南高等实业学堂的矿科和路科。原湖南医科大学的前身为1914年创建的湘雅医学专门学校,是我国创办最早的西医高等学校之一。中南大学秉承百年办学积淀,顺应中国高等教育体制改革大势,弘扬以“知行合一、经世致用”为核心的大学精神,力行“向善、求真、唯美、有容”的校风,坚持自身办学特色,服务国家和社会重大需求,团结奋进,改革创新,追求卓越,综合实力和整体水平大幅提升。


\newpage


