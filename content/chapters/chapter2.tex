%!TEX root = ../../csuthesis_main.tex
\chapter{图表示例}

\section{图片与布局}

\subsection{插图}

图片可以通过\cs{includegraphics}指令插入,我们建议模板使用者将文章所需插入的图片源问卷放置在 images 目录中,
另外,矢量图片应使用PDF格式,位图照片则应使用JPG格式(LaTeX不支持TIFF格式)。具有透明背景的栅格图可以使用PNG格式。
模板已经配置了相对路径,所以在文中插图片时,直接用 images 目录下的相对路径即可,比如 images/csu.png ,
在正文中插入只需要\cs{includegraphics{csu.png}},不需要再增加前缀。

下面是一个简单的插图示例。

\begin{figure}[hbt]
    \centering
    \includegraphics[width=0.3\linewidth]{csu_logo_blue.png}
    \caption{插图示例}
    \label{f.example}
\end{figure}


如果一个图由多个分图(子图)组成,应通过(a),(b),(c)进行标识并附注在分图(子图下方)。

\subsection{横向布局}

模板提供常见的图片布局,比如单图布局\ref{f.example},另外还有横排布局如下:

\begin{figure}[!htb]
    \centering
    \begin{subfigure}[t]{0.24\linewidth}
        \captionsetup{justification=centering}
        \begin{minipage}[b]{1\linewidth}
        \includegraphics[width=1\linewidth]{csu_logo_blue.png}
        \caption{test}
        \end{minipage}
    \end{subfigure}
    \begin{subfigure}[t]{0.24\linewidth}
        \captionsetup{justification=centering}
        \begin{minipage}[b]{1\linewidth}
        \includegraphics[width=1\linewidth]{csu_logo_black.png}
        \caption{test}
        \end{minipage}
    \end{subfigure}
    \begin{subfigure}[t]{0.24\linewidth}
        \captionsetup{justification=centering}
        \begin{minipage}[b]{1\linewidth}
        \includegraphics[width=1\linewidth]{csu_logo_blue.png}
        \caption{test}
        \end{minipage}
    \end{subfigure}
    \begin{subfigure}[t]{0.24\linewidth}
        \captionsetup{justification=centering}
        \begin{minipage}[b]{1\linewidth}
        \includegraphics[width=1\linewidth]{csu_logo_black.png}
        \caption{test}
        \end{minipage}
    \end{subfigure}
    \caption{图片横排布局示例}
    \label{f.row}
\end{figure}

\section{纵向布局}

纵向布局如图\ref{f.col}

\begin{figure}[!htb]
    \centering
    \begin{subfigure}[t]{0.15\linewidth}
        \captionsetup{justification=centering} %ugly hacks
        \begin{minipage}[b]{1\linewidth}
        \includegraphics[width=1\linewidth]{csu_logo_blue.png}
        \caption{test}
        \end{minipage}
    \end{subfigure}\\
    \begin{subfigure}[t]{0.15\linewidth}
        \captionsetup{justification=centering} %ugly hacks
        \begin{minipage}[b]{1\linewidth}
        \includegraphics[width=1\linewidth]{csu_logo_black.png}
        \caption{test}
        \end{minipage}
    \end{subfigure}
    \caption{图片纵向布局示例}
    \label{f.col}
\end{figure}

\section{竖排多图横排布局}

\begin{figure}[!htb]
    \centering
    \begin{subfigure}[t]{0.13\linewidth}
        \captionsetup{justification=centering} 
        \begin{minipage}[b]{1\linewidth}
        \includegraphics[width=1\linewidth]{csu_logo_blue.png} 
        \vspace{-1ex} \vfill
        \includegraphics[width=1\linewidth]{csu_logo_black.png}
        \caption{aaa}
        \end{minipage}
    \end{subfigure}
    \begin{subfigure}[t]{0.13\linewidth}
        \captionsetup{justification=centering} 
        \begin{minipage}[b]{1\linewidth}
        \includegraphics[width=1\linewidth]{csu_logo_black.png} 
        \vspace{-1ex} \vfill
        \includegraphics[width=1\linewidth]{csu_logo_blue.png}
        \caption{bbb}
        \end{minipage}
    \end{subfigure}
    \caption{图片竖排多图横排布局}
    \label{f.csu_col_row}
\end{figure}

竖排多图横排布局如图\ref{f.csu_col_row}所示。注意看(a)、(b)编号与图关系


\section{横排多图竖排布局}

中南大学由原湖南医科大学、长沙铁道学院与中南工业大学于2000年4月合并组建而成。原中南工业大学的前身为创建于1952年的中南矿冶学院,原长沙铁道学院的前身为创建于1953年的中南土木建筑学院,两校的主体学科最早溯源于1903年创办的湖南高等实业学堂的矿科和路科。原湖南医科大学的前身为1914年创建的湘雅医学专门学校,是我国创办最早的西医高等学校之一。中南大学秉承百年办学积淀,顺应中国高等教育体制改革大势,弘扬以“知行合一、经世致用”为核心的大学精神,力行“向善、求真、唯美、有容”的校风,坚持自身办学特色,服务国家和社会重大需求,团结奋进,改革创新,追求卓越,综合实力和整体水平大幅提升。

\begin{figure}[!htb]
    \centering
    \begin{subfigure}[t]{0.3\linewidth}
        \captionsetup{justification=centering} 
        \begin{minipage}[b]{1\linewidth}
        \includegraphics[width=0.45\linewidth]{csu_logo_blue.png}
        \includegraphics[width=0.45\linewidth]{csu_logo_black.png}
        \caption{}
        \end{minipage}
    \end{subfigure}\\
    \begin{subfigure}[t]{0.3\linewidth}
        \captionsetup{justification=centering} 
        \begin{minipage}[b]{1\linewidth}
        \includegraphics[width=0.45\linewidth]{csu_logo_black.png}
        \includegraphics[width=0.45\linewidth]{csu_logo_blue.png}
        \caption{}
        \end{minipage}
    \end{subfigure}
    \caption{图片横排多图竖排布局}
    \label{f.csu_row_col}
\end{figure}

横排多图竖排布局如图\ref{f.csu_row_col}所示。注意看(a)、(b)编号与图关系。

\section{2x2图片布局}

\section{ 2x2 布局}

\begin{figure}[!htb]
    \centering
    \begin{subfigure}[t]{0.3\linewidth}
        \captionsetup{justification=centering}
        \begin{minipage}[b]{1\linewidth}
            \centering
            \includegraphics[width=0.45\linewidth]{csu_logo_blue.png}
            \caption{}
        \end{minipage}
    \end{subfigure}
    \hspace{-5em}
    \begin{subfigure}[t]{0.3\linewidth}
        \captionsetup{justification=centering}
        \begin{minipage}[b]{1\linewidth}
            \centering
            \includegraphics[width=0.45\linewidth]{csu_logo_black.png}
            \caption{}
        \end{minipage}
    \end{subfigure}\\
    \begin{subfigure}[t]{0.3\linewidth}
        \captionsetup{justification=centering}
        \begin{minipage}[b]{1\linewidth}
            \centering
            \includegraphics[width=0.45\linewidth]{csu_logo_blue.png}
            \caption{}
        \end{minipage}
    \end{subfigure}
    \hspace{-5em}
    \begin{subfigure}[t]{0.3\linewidth}
        \captionsetup{justification=centering}
        \begin{minipage}[b]{1\linewidth}
            \centering
            \includegraphics[width=0.45\linewidth]{csu_logo_black.png}
            \caption{}
        \end{minipage}
    \end{subfigure}
    \caption{图片2x2布局}
    \label{f.csu_2x2}
\end{figure}

\newpage
