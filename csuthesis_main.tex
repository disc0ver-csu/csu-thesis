%%%%%%%%%%%%%%%%%%%%%%%%%%%%%%%%%%%%%%%%%%%%%%%%%%
% 载入模版
%
% 载入csuthesis.cls文件定义的模板
% 支持选项,forprint(黑白打印模式)
%%%%%%%%%%%%%%%%%%%%%%%%%%%%%%%%%%%%%%%%%%%%%%%%%%
\documentclass[forprint]{csuthesis}

\addbibresource{CSUthesis_main.bib}

%%%%%%%%%%%%%%%%%%%%%%%%%%%%%%%%%%%%%%%%%%%%%%%%%%
% 基本信息
%
% 用户自行输入标题、作者等基本信息
% 都存储在\content\info.tex文件中
%%%%%%%%%%%%%%%%%%%%%%%%%%%%%%%%%%%%%%%%%%%%%%%%%%
% 文章信息
\titlecn{中南大学研究生学位论文LaTeX模板}
\titleen{LaTeX Template of Postgraduate Thesis of \\Central South University}

\priormajor{计算机科学与技术}
\minormajor{计算机应用技术}
\interestmajor{旁门左道}
\author{郭大侠}
\supervisor{我自己\ 教授}
\subsupervisor{}
\department{计算机学院}
\studentid{144601044}
\thesisdate{year=2019,month=5}



\clcnumber{TP391} 				% 中图分类号 Chinese Library Classification
\schoolcode{10533}			% 学校代码
\udc{004.9}						% UDC
\academiccategory{学术学位}	% 学术类别


\newif \ifblindreview % 条件语句,是否是盲审版本
% \blindreviewtrue
\blindreviewfalse


\begin{document}
%%%%%%%%%%%%%%%%%%%%%%%%%%%%%%%%%%%%%%%%%%%%%%%%%%
% 封面绘制
%
% 1.5版本重新编写了封面绘制宏,并用latex使用者更习惯的
% \maketitle代替之前的\makecoverpage
%%%%%%%%%%%%%%%%%%%%%%%%%%%%%%%%%%%%%%%%%%%%%%%%%%
\maketitle

% 启用大罗马字母进行编号
\frontmatter
% 设置页眉和页脚


%%%%%%%%%%%%%%%%%%%%%%%%%%%%%%%%%%%%%%%%%%%%%%%%%%
% 中文摘要
%
% 存储在\content\abstractzh.tex文件中
%%%%%%%%%%%%%%%%%%%%%%%%%%%%%%%%%%%%%%%%%%%%%%%%%%
\include{content/abstractzh}

%%%%%%%%%%%%%%%%%%%%%%%%%%%%%%%%%%%%%%%%%%%%%%%%%%
% 英文摘要
%
% 存储在\content\abstracten.tex文件中
%%%%%%%%%%%%%%%%%%%%%%%%%%%%%%%%%%%%%%%%%%%%%%%%%%
%!TEX root = ../csuthesis_main.tex
\keywordsen{CSU\ \ LaTeX\ \ Template}
\begin{abstracten}

LaTeX can be compiled into a pdf of uniform format using the set template.At present, most domestic publishers and universities still use word. Because of its powerful function and flexibility, when faced with fixed-form papers by novices, simple matters such as typesetting, numbering, and reference documents will bring many difficulties and troubles. For some problems that need to be modified throughout, to achieve the efficiency of LaTeX, it requires a high level of skill for word users.

In order to focus on the writing of papers, many international journals and universities support the writing and submission of LaTeX. Novices don't need to care about formatting issues. They only need to use a few symbolic labels step by step to get the documents that meet the requirements. And when you need to modify the entire format, you can directly recompile the template file by replacing or modifying the template file. This is incredible for the word novice to use the word.

The purpose of this project is to create a TeX template that meets the specifications of the graduate degree thesis (PhD) of Central South University, and to address the pain points of format adjustment during the dissertation writing.
\end{abstracten}


%%%%%%%%%%%%%%%%%%%%%%%%%%%%%%%%%%%%%%%%%%%%%%%%%%
% 目录
%
% 使用重定义的tableofcontents宏绘制目录
% 满足学校的样式要求
%%%%%%%%%%%%%%%%%%%%%%%%%%%%%%%%%%%%%%%%%%%%%%%%%%
\tableofcontents


% 启用数字编号,改为第 x 页  共 x 页格式
\mainmatter

%%%%%%%%%%%%%%%%%%%%%%%%%%%%%%%%%%%%%%%%%%%%%%%%%%
% 正文
%
% 存储在\content\content.tex文件中
%%%%%%%%%%%%%%%%%%%%%%%%%%%%%%%%%%%%%%%%%%%%%%%%%%
% 正文
% 论文正文是主体,主体部分应从另页右页开始,每一章应另起页。一般由序号标题、文字叙述、图、表格和公式等五个部分构成。
\section{绪论}
\subsection{研究背景与意义}

Word不难用,但是想用得漂亮还得费一番功夫。

对于小白来说,(注意!!是对于小白来说,不要跟我杠!!!我就是word小白,高端玩法玩不动):

插入个图片,下面的说明文字是不是插入文本框?那文本框要不要跟图片“组合”?,是不是直接圈没法圈起来?因为图片要变成浮动格式,和文本框绑定后再改回嵌入格式,你说蛋疼不蛋疼?不组合?那有一定概率发生你的图片在上一页,描述文字在下一页。呵呵。

插入参考文献,手动编辑?我的天哪,一百多个文献,中间插一个,怎么改序号? 
很好,可以用交叉引用,一个个编辑文献格式?
很好,可以用endnote或者noteexpress的插入功能,你有没有发现插入是个宏,ctrl+z的时候会烦死你啊……
叮叮!让我们祭出LaTeX!!,有bibliography,一个$\\cite$包打天下!不要太爽。

插入公式,对word小白来说,公式居中编号靠右就是一道百度搜索能力过滤器。

word里编辑三线表,啊烦躁。

等等等等……

让我们,专心写论文好不好?

爱你们。


\subsection{主要研究工作}
我堂堂双一流高校竟然没有官方研究生论文LaTeX模板!!!虽然我LaTeX水平也很水……但是通过大量debug也勉强给大家凑出来一个格式绝对标准的LaTeX模板,模板代码丑就丑吧,能用就行。写了大量注释,有一点LaTeX基础就可以根据自己需要修改CSUthesis.cls文件。

(1) 提供图片插入示例。

(2) 提供表格插入示例。

(3) 提供公式插入示例。

(4) 提供参考文献插入示例。

\subsection{论文组织结构}

全文内容共六章,具体内容组织如下:

第一章为绪论。

第二章为图片插入示例。

第三章为表格插入示例。

第四章为公式插入示例。

第五章为参考文献插入示例。

第六章总结与展望,总结了本文的主要工作,展望了下一阶段的研究方向。

\newpage

\section{图像布局}
\label{sec.figure}
\textbf{按学校格式要求,每个子图的小标(a)、(b)、(c)等在【左上角】。}

\subsection{单图布局}

\lipsum

\textbf{单图布局如图\ref{F.csu_single}所示。}

\begin{figure}[hbt]
\centering
\includegraphics[width=0.5\textwidth]{csu.png}
\caption{单图布局示例}
\label{F.csu_single}
\end{figure}

\subsection{横排布局}

\textbf{横排布局如图\ref{F.csu_row}所示。}

\begin{figure}[!htb]
    \centering
    \begin{subfigure}[t]{0.24\linewidth}
        \caption{}
        \begin{minipage}[b]{1\linewidth}
        \includegraphics[width=1\linewidth]{csu.png}
        \end{minipage}
    \end{subfigure}
    \begin{subfigure}[t]{0.24\linewidth}
        \caption{}
        \begin{minipage}[b]{1\linewidth}
        \includegraphics[width=1\linewidth]{csu.png}
        \end{minipage}
    \end{subfigure}
    \begin{subfigure}[t]{0.24\linewidth}
        \caption{}
        \begin{minipage}[b]{1\linewidth}
        \includegraphics[width=1\linewidth]{csu.png}
        \end{minipage}
    \end{subfigure}
    \begin{subfigure}[t]{0.24\linewidth}
        \caption{}
        \begin{minipage}[b]{1\linewidth}
        \includegraphics[width=1\linewidth]{csu.png}
        \end{minipage}
    \end{subfigure}
    \caption{横排布局示例}
    \label{F.csu_row}
\end{figure}

\lipsum

\subsection{竖排布局}
\textbf{竖排布局如图\ref{F.csu_col}所示。}

\begin{figure}[!htb]
    \centering
    \begin{subfigure}[t]{0.15\linewidth}
        \caption{}
        \begin{minipage}[b]{1\linewidth}
        \includegraphics[width=1\linewidth]{csu.png}
        \end{minipage}
    \end{subfigure}\\
    \begin{subfigure}[t]{0.15\linewidth}
        \caption{}
        \begin{minipage}[b]{1\linewidth}
        \includegraphics[width=1\linewidth]{csu.png}
        \end{minipage}
    \end{subfigure}
    \caption{竖排布局示例}
    \label{F.csu_col}
\end{figure}

\lipsum

\subsection{竖排多图横排布局}

\begin{figure}[!htb]
    \centering
    \begin{subfigure}[t]{0.13\linewidth}
        \caption{}
        \begin{minipage}[b]{1\linewidth}
        \includegraphics[width=1\linewidth]{csu.png} \vspace{-1ex} \vfill 
        \includegraphics[width=1\linewidth]{csu.png}
        \end{minipage}
    \end{subfigure}
    \begin{subfigure}[t]{0.13\linewidth}
        \caption{}
        \begin{minipage}[b]{1\linewidth}
        \includegraphics[width=1\linewidth]{csu.png} \vspace{-1ex} \vfill 
        \includegraphics[width=1\linewidth]{csu.png}
        \end{minipage}
    \end{subfigure}
    \caption{竖排多图横排布局}
    \label{F.csu_col_row}
\end{figure}

\textbf{竖排多图横排布局如图\ref{F.csu_col_row}所示。注意看(a)、(b)编号与图关系。}


\subsection{横排多图竖排布局}

\lipsum

\begin{figure}[!htb]
    \centering
    \begin{subfigure}[t]{0.3\linewidth}
        \caption{}
        \begin{minipage}[b]{1\linewidth}
        \includegraphics[width=0.45\linewidth]{csu.png}
        \includegraphics[width=0.45\linewidth]{csu.png}
        \end{minipage}
    \end{subfigure}\\
    \begin{subfigure}[t]{0.3\linewidth}
        \caption{}
        \begin{minipage}[b]{1\linewidth}
        \includegraphics[width=0.45\linewidth]{csu.png}
        \includegraphics[width=0.45\linewidth]{csu.png}
        \end{minipage}
    \end{subfigure}
    \caption{横排多图竖排布局}
    \label{F.csu_row_col}
\end{figure}

\textbf{横排多图竖排布局如图\ref{F.csu_row_col}所示。注意看(a)、(b)编号与图关系。}

\subsection{本章小结}
本章示例图片布局。

\newpage


\section{表格插入示例}

\begin{table}[htb]
  \centering
  \caption{学校文件里对表格的要求不是很高,不过按照学术论文的一般规范,表格为三线表。}
  \label{T.example}
  \begin{tabular}{llllll}
  \hline
   & A  & B  & C  & D  & E \\
  \hline
1 	& 212 & 414 & 4 		& 23 & fgw	\\
2 	& 212 & 414 & v 		& 23 & fgw	\\
3 	& 212 & 414 & vfwe		& 23 & 嗯	\\
4 	& 212 & 414 & 4fwe		& 23 & 嗯	\\
5 	& af2 & 4vx & 4 		& 23 & fgw	\\
6 	& af2 & 4vx & 4 		& 23 & fgw	\\
7 	& 212 & 414 & 4 		& 23 & fgw	\\

\hline{}
\end{tabular}
\end{table}

\textbf{表格如表\ref{T.example}所示,latex表格技巧很多,这里不再详细介绍。}

\lipsum

\newpage

\section{公式插入示例}

\lipsum

\textbf{公式插入示例如公式(\ref{E.example})所示。}

\begin{equation}
\gamma_{x}=
\left\{
  \begin{array}{lr}
  0, & {\rm if}~~\;|x| \leq \delta \\
  x, & {\rm otherwise}
  \end{array}
\right.
\label{E.example}
\end{equation}


\newpage

\section{参考文献插入示例}

LaTeX\cite{lamport1994latex}插入参考文献最方便的方式是使用bibliography\cite{pritchard1969statistical},大多数出版商的论文页面都会有导出bib格式参考文献的链接,把每个文献的bib放入``thesis-references'',然后用bibkey即可插入参考文献。

\lipsum

\newpage


\section{总结与展望}

XX的XX都存在XX,所以我们XX,本章总结XX。

\subsection{本文工作总结}
在总结和分析已有XX的理论基础上,本文对XX进行了XX,主要工作如下:

(1)图片插入布局,如第\ref{sec.figure}章所示。

(2)XXXXXXXXXX

(3)XXXXXXXXXX

(4)XXXXXXXXXX

\subsection{工作展望}
本课题针对XX,鉴于XXX,对XX进行了提高,但是XXX,所以有如下XX:

(1)目前XX虽然XX,但是XX仍然XX,所以XX仍然是一个值得XX的问题。

(2)随着XX,XX具有XX的问题,仍值得进一步XX。

(3)本课题在XX有了XX,但是XX的XX还存在XX,所以XX。


\newpage





% https://www.zhihu.com/question/29413517/answer/44358389 %
% 说明如下:
% secnumdepth 这个计数器是 LaTeX 标准文档类用来控制章节编号深度的。在 article 中,这个计数器的值默认是 3,对应的章节命令是 \subsubsection。也就是说,默认情况下,article 将会对 \subsubsection 及其之上的所有章节标题进行编号,也就是 \part, \section, \subsection, \subsubsection。LaTeX 标准文档类中,最大的标题是 \part。它在 book 和 report 类中的层级是「-1」,在 article 类中的层级是「0」。这里,我们在调用 \appendix 的时候将计数器设置为 -2,因此所有的章节命令都不会编号了。不过,一般还是会保留 \part 的编号的。所以在实际使用中,将它设置为 0 就可以了。

% 在修改过程中请注意不要破环命令的完整性

% \renewcommand\appendix{\setcounter{secnumdepth}{-2}}
% \appendix

% % 主文件有代码去掉页眉章节编号的“.”,但这会因为bug导致无编号章节显示一个错误编号,所以这里在无编号章节之前再次重定义sectionmark。
% \renewcommand{\sectionmark}[1]{\markright{#1}}


%%%%%%%%%%%%%%%%%%%%%%%%%%%%%%%%%%%%%%%%%%%%%%%%%%
% 致谢
%
% 存储在\content\acknowledgements.tex文件中
% 根据本科生院的要求,致谢应该在参考文献的前面,不编章号,而附录应该位于参考文献后。
%%%%%%%%%%%%%%%%%%%%%%%%%%%%%%%%%%%%%%%%%%%%%%%%%%
%!TEX root = ../csuthesis_main.tex
\begin{acknowledgements} 
    % 无章节编号

    感谢做出第一版中南大学博士学位论文模板的郭大侠。

    感谢16级的姜析阅学长,@Blurry,@baoxuan等对模板做出的贡献。

    感谢上海交大学位论文Latex模板、武汉大学学位论文模板的制作者们提供的经验!

    感谢William Wang 同学对模板移植做出的巨大贡献!

    感谢@weijianwen 学长一直以来的开发和维护工作!

    感谢所有为模板贡献过代码的同学们!
\end{acknowledgements}


%%%%%%%%%%%%%%%%%%%%%%%%%%%%%%%%%%%%%%%%%%%%%%%%%%
% 参考文献
%
% 存储在\content\acknowledgements.tex文件中
% 根据本科生院的要求,致谢应该在参考文献的前面,不编章号,而附录应该位于参考文献后。
% 有待修复
%%%%%%%%%%%%%%%%%%%%%%%%%%%%%%%%%%%%%%%%%%%%%%%%%%
% \section{参考文献} % bibliography会自动显示参考文献四个字
\addcontentsline{toc}{chapter}{参考文献} % 由于参考文献不是chapter,这句把参考文献加入目录
% \nocite{*} % 该命令用于显示全部参考文献,即使文中没引用
% cls文件中已经引入package,这里不需要调用 \bibliographystyle 了。
% \bibliographystyle{gbt7714-2005}
% \bibliography{thesis-references}
\printbibliography

\newpage


%%%%%%%%%%%%%%%%%%%%%%%%%%%%%%%%%%%%%%%%%%%%%%%%%%
% 附录部分
%
% 根据学校要求,正文中不应出现长篇幅的代码段或公式推证
% 应单独放置在正文后的附录部分
%%%%%%%%%%%%%%%%%%%%%%%%%%%%%%%%%%%%%%%%%%%%%%%%%%
%!TEX root = ../csuthesis_main.tex
% \begin{appendixs} % 无章节编号
\chapter{附录代码}

附录部分用于存放这里用来存放不适合放置在正文的大篇幅内容、典型如代码、图纸、完整数学证明过程等内容。

\section{堆溢出检测算法}

\begin{algorithm}[h]
    \caption{堆溢出检测算法}\label{alg:ovf}
    \begin{algorithmic}[1]
        \IF {$\beta \in \mathbb{N^{*}} \land \Delta_\beta = \Delta_{\beta - 1} \land \beta < S$}
            \STATE 正常写入
        \ELSIF {$\beta \in \mathbb{N^{*}} \land \Delta_\beta \neq \Delta_{\beta - 1} \land \beta \geq S$}
            \STATE 发生堆溢出
        \ENDIF
    \end{algorithmic}
\end{algorithm}

\section{KMP算法C++描述}

% \begin{minted}[linenos]{c}
\begin{lstlisting}
    const int maxn=2e5+5; 
    int nt[maxn];
    int aa[maxn],bb[maxn];
    int a[maxn],b[maxn];
    int n;
    //参数为模板串和next数组
    //字符串均从下标0开始
    void kmpGetNext(int *s,int *Next)
    {
        Next[0]=0;
    //    int len=strlen(s);
        for(int i=1,j=0;i<n;i++)
        {
            while(j&&s[i]!=s[j]) j=Next[j];
            if(s[i]==s[j]) j++;
            Next[i+1]=j;
        }
    //    Next[len]=0;
    }
    int kmp(int *ss,int *s,int *Next)
    {
        kmpGetNext(s,Next);
    //  调试输出Next数组
    //	int len=strlen(s);
    //	for(int i=0;i<=n;i++)
    //		cout<<Next[i]<<" ";
    //	cout<<endl; 
    
    //    int ans=0;
    //    int len1=strlen(ss);
    //    int len2=strlen(s);
        for(int i=0,j=0;i<2*n;i++)  //倍长 
        {
            while(j&&ss[i%n]!=s[j])j=Next[j];
            if(ss[i%n]==s[j]) j++;
            if(j==n){
                return 1;
            }
               
        }
        return 0;
    }
    int main(void)
    {
        while(cin>>n)
        {
            memset(a,0,sizeof(a));
            memset(b,0,sizeof(b));
            rep(i,0,n) cin>>aa[i];
            rep(i,0,n) cin>>bb[i];
            sort(aa,aa+n);
            sort(bb,bb+n);
            rep(i,0,n-1){
                a[i]=aa[i+1]-aa[i];
                b[i]=bb[i+1]-bb[i];
            }
            a[n-1]=360000+aa[0]-aa[n-1];
    //		rep(i,0,n) cout<<a[i]<<" ";
    //		cout<<endl;
            b[n-1]=360000+bb[0]-bb[n-1];
    //		rep(i,0,n) cout<<b[i]<<" ";
    //		cout<<endl;
            if(kmp(a,b,nt))
                cout<<"possible"<<endl;
            else cout<<"impossible"<<endl;
        }
        return 0;
    }
\end{lstlisting}
% \end{minted}

\chapter{康托尔辩辞录:数学的自由与制约}

(录自康托尔:《一般集合论基础》,1883)

数学在其发展中是完全自由的,它只受下述自明的关注所制约,即它的概念既要内在地不存在矛盾,还要参与确定与此前形成的,已经存在着地和已被证明地概念之关系(借助定义贯串起来)。特别地,在引入新数时,数学只遵循:在给出它们地定义时使之具有某种确定性,并且在某些情况下,使之与老数有某种关系,在特定地场合中这种关系一定会使它们(新数和老数)互相区别开来,只要一个数满足这些条件,数学只能而且必须把它看作是存在的和实在的东西,这正是我……关于为什么必须把有理数、无理数和复数看作与有限正整数一样是实在的所建议的理由。

我相信,没有必要害怕,许多人是害怕,这些原则含有对于科学的危险,一方面,实行造出新数的自由必须服从所设计的条件,但这些条件给任意性留下的活动空间是非常小的。而且,每一数学概念在其自身之中也带有必要的矫正物;如果它没有收获也不合适(它的无用很快就会表明这一点),那么它将由于没有成功而被丢弃。另一方面,在我看来,对于数学研究工作的任何多余的限制只会随之而带来更大的危险,由于实际上并没有任何理由可说明它是由科学的本质推断出来的,它的危险就更大了,而数学的本质恰恰在于它的自由。

如果高斯、柯西、阿贝尔、雅可比、狄利克雷、魏尔斯特拉斯、埃尔米特和黎曼总是被束缚而拿他们的新想法去臣服于形而上学的控制,那么,我们今日就不可能为现代函数论的雄伟建筑而高兴,现代函数论的设计和矗立是完全自由的,毫无短视的瞬间目的……。如果福克斯、庞加莱和其他许多杰出的智者受外来影响所包围和限制,我们就会见不到他们带给微分方程论的巨大的推动,还有,如果枯莫尔不是斗胆地(大有仿效者)把所谓的“理想”数引入数论,我们今天也无从去羡慕钦佩克罗内克和戴德金在代数和算术上十分重要和杰出的工作。

因此,如已说明的,数学是要脱离形而上学的桎梏而完全自由地发展 \dots



% \end{appendixs}



%%%%%%%%%%%%%%%%%%%%%%%%%%%%%%%%%%%%%%%%%%%%%%%%%%
% 科研成果
%
% TODO: 硕士博士学位论文需添加
%%%%%%%%%%%%%%%%%%%%%%%%%%%%%%%%%%%%%%%%%%%%%%%%%%

\end{document}
