% !Mode:: "TeX:UTF-8"
%%%%%%%%-------杂七杂八的话--------%%%%%%%%


\chapter{杂七杂八的话}

\section{Readme}

模板文件的结构, 如下表所示:
\begin{table}[ht]\centering
  \begin{tabular}{r|r|l}
    \hline\hline
    \multicolumn{2}{l|}{Bachelor-template.tex } & 主文档. 在其中填写正文.                                      \\ \hline
                                                & frontmatter.tex                      & 郑重声明、中英文摘要. \\ \cline{2-3}
    \raisebox{1em}{includefile 文件夹}          & backmatter.tex                       & 致谢.                 \\ \hline
    \multicolumn{2}{l|}{figures 文件夹}         & 存放图片文件.                                                \\ \hline
    \multicolumn{2}{l|}{WHUBachelor.cls }       & 定义文档格式的 class file. 不可删除.                         \\ \hline\hline
  \end{tabular}
\end{table}

无需也不要改变、移动上述文档的位置.

如果不习惯用~\verb|\include{ }|~的方式加入``子文档'', 当然可以把它们合并在主文档, 成为一个文档.
({\kaishu 但是这样并不会给我们带来方便.})

利用~WinEdt~的~Project tree, 可以方便地管理这些文件:
\begin{itemize}
  \item 点击~WinEdt~窗口的~Project Tree~按键;
  \item 再点击~WinEdt~窗口的~Set Main File~按键;
\end{itemize}
接下来的管理, 已经清楚地展示在跳出的窗口中了. 再去处理其他的文件时, 还要点击~WinEdt~窗口的~Remove Main File~按键.


\section{更新记录}
2016 年 06 月更新: 正文字体为小四号; 英文字体为 Times New Roman; 修订图表标题的字体、字号; 修订目录的字号; 修订附录章节编号的问题.
非常感谢武汉大学数学与统计学院 2012 级张仕俊、林颖倩、宋俍辰等同学.

2016 年 05 月更新: 参考文献加到目录. 感谢武汉大学经济与管理学院的郑中天同学. [上次修订使用的版本有误, 非常抱歉.]

2016 年 02 月更新: 调整为适应 TeX Live 2015 的版本.

2014 年 06 月更新: 修改章节标题、声明标题、图表标题的字体和大小. 再次感谢孙启航同学.

2014 年 05 月更新: 参考文献加到目录. 感谢武汉大学计算机学院孙启航同学、数学与统计学院李振坤同学指出这个纰漏.

2013 年 12 月更新: 加上英文封面. 教务部的写作规范中的附例, 并没有英文封面. 但是遇到很多同学说要加上.


\section{字体调节}

\begin{tabular}{ll}
  \verb|\songti| & {\songti 宋体}   \\
  \verb|\heiti| & {\heiti 黑体}    \\
  \verb|\fangsong| & {\fangsong 仿宋} \\
  \verb|\kaishu| & {\kaishu 楷书}
\end{tabular}


\section{字号调节}
字号命令: \verb|\zihao| \index{zihao}

\begin{tabular}{ll}
  \verb|\zihao{0}|  & \zihao{0}  初号字 English \\
  \verb|\zihao{-0}|  & \zihao{-0} 小初号 English \\
  \verb|\zihao{1} |  & \zihao{1}  一号字 English \\
  \verb|\zihao{-1}| & \zihao{-1} 小一号 English \\
  \verb|\zihao{2} | & \zihao{2}  二号字 English \\
  \verb|\zihao{-2}| & \zihao{-2} 小二号 English \\
  \verb|\zihao{3} | & \zihao{3}  三号字 English \\
  \verb|\zihao{-3}| & \zihao{-3} 小三号 English \\
  \verb|\zihao{4} | & \zihao{4}  四号字 English \\
  \verb|\zihao{-4}| & \zihao{-4} 小四号 English \\
  \verb|\zihao{5} | & \zihao{5}  五号字 English \\
  \verb|\zihao{-5}| & \zihao{-5} 小五号 English \\
  \verb|\zihao{6} | & \zihao{6}  六号字 English \\
  \verb|\zihao{-6}| & \zihao{-6} 小六号 English \\
  \verb|\zihao{7} | & \zihao{7}  七号字 English \\
  \verb|\zihao{8} | & \zihao{8}  八号字 English \\
\end{tabular}

\section{已加入的常用宏包}

\begin{description}
  %  \item[amsmath,amssymb]
  \item[cite]  参考文献引用, 得到形如 [3-7] 的样式.
  \item[color,xcolor]  支持彩色.
  \item[enumerate]  方便自由选择 enumerate 环境的编号方式. 比如

        \verb|\begin{enumerate}[(a)]| 得到形如 (a), (b), (c) 的编号.


        \verb|\begin{enumerate}[i)]| 得到形如 i), ii), iii) 的编号.

\end{description}

另外要说明的是,  itemize, enumerate, description 这三种 list 环境, 已经调节了其间距和缩进,
以符合中文书写的习惯.

\section{标点符号的问题}

建议使用半角的标点符号, 后边再键入一个空格. 特别是在英文书写中要注意此问题!

双引号是由两个左单引号、两个右单引号构成的: \verb|``  ''|. 左单引号在键盘上数字~1 的左边.

但是, 无论您偏向于全角或半角, 强烈建议您使用实心的句号, 只要您书写的是自然科学的文章.
原因可能是因为, 比如使用全角句号的句子结尾处的``$x$。''容易误为数学式~$x_0$(\verb|$x_0$|)吧.
