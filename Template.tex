% !Mode:: "TeX:UTF-8"
%% 请使用 XeLaTeX 编译本文.
% \documentclass{EdwardEssay}% 选项 forprint: 交付打印时添加, 避免彩色链接字迹打印偏淡. 即使用下一行:
\documentclass[forprint]{EdwardEssay}

\begin{document}
%%%%%%% 下面的内容, 据实填空.

% \miji{ }                                      % 密级. 没有就空着.
\StudentNumber{0000000000} % 填写自己的学号
\title{你的题目}
\Etitle{Your Title} % 英文题目
\author{张三}                            % 作者名字
\Eauthor{San Zhang}            %作者英文名
\Csupervisor{X \quad 教授}        %指导教师中文名、职称
\Esupervisor{Prof.~X}     %指导教师英文名、职称
% \Cmajor{计算机科学与技术}                  % 专业中文名
\Cmajor{计科1707}                  % 由于学校习惯 这里为专业班级 如计科1707
\Emajor{Computing Science and Technology}% 专业英文名
\Cschoolname{计算机学院}          % 学院名
\Eschoolname{School of Computer Science and Engineering} %学院英文名. 不确定的话, 请看一下自己学院的网页上是怎么写的. 别搞错了!
\date{二〇一九年十二月}                    % 日期, 要注意和英文日期一致!!
\Edate{December, 2019}                       % 英文封面日期

%-----------------------------------------------------------------------------
\pdfbookmark[0]{封面}{title}         % 封面页加到 pdf 书签
\maketitle
\frontmatter
% \pagenumbering{Roman}              % 正文之前的页码用大写罗马字母编号.
%-----------------------------------------------------------------------------
% !Mode:: "TeX:UTF-8"

%%% 此部分需要自行填写: (1) 中文摘要及关键词 (2) 英文摘要及关键词
%%%%%%%%%
%%% -------------  英文封面 (无需改动)-------------   %%%
%%%%%%%%%
\thispagestyle{empty}
\renewcommand{\baselinestretch}{1.5}  %下文的行距  %到正文部分
\vspace*{0.5cm}
\begin{center}
{\Large \bf STUDENT ESSAY \\[1ex] OF CENTRAL SOUTH UNIVERSITY }
\end{center}
\vspace{2.5cm}
\begin{center}{\zihao{2} \the\Etitle \par}\end{center}

\vfill

\begin{center}
\zihao{4}
\begin{tabular}{ r l }
 School (Department): & {\sc \the\Eschoolname}\\
  Major:          &   {\sc\the\Emajor}  \\
 Candidate:      &  {\sc \the\Eauthor}      \\
 StudentID:      &  {\sc \the\StudentNumber} \\
 Supervisor:     &  {\sc \the\Esupervisor}
\end{tabular}

\vspace*{2cm}
\begin{center}
   \ifprint % 文档打印, 使用黑白校徽.
  % \includegraphics[height=4cm]{whu.eps}       %%  黑白的.
  \includegraphics[height=4cm]{csu_wb_logo.png}
  \else
  % \includegraphics[height=4cm]{whulogo.eps} %%  彩色的.
  \includegraphics[height=4cm]{csu_logo.png}
  \fi
\end{center}


\zihao{-2}
%\the\Schoolname\\
{\sc Central South University}

\vspace*{1.0cm}

\the\Edate

\end{center}
%%% 郑重声明部分无需改动

%%%---- 郑重声明 (无需改动)------------------------------------%
\newpage
% \vspace*{20pt}
% \begin{center}{\ziju{0.8}\textbf{\songti\zihao{2} 郑重声明}}\end{center}
% \par\vspace*{30pt}
% \renewcommand{\baselinestretch}{2}

% {\zihao{4}%

% 本人呈交的学位论文, 是在导师的指导下, 独立进行研究工作所取得的成果,
% 所有数据、图片资料真实可靠. 尽我所知, 除文中已经注明引用的内容外,
% 本学位论文的研究成果不包含他人享有著作权的内容.
% 对本论文所涉及的研究工作做出贡献的其他个人和集体,
% 均已在文中以明确的方式标明. 本学位论文的知识产权归属于培养单位.\\[2cm]

% \hspace*{1cm}本人签名: $\underline{\hspace{3.5cm}}$
% \hspace{2cm}日期: $\underline{\hspace{3.5cm}}$\hfill\par}
% %------------------------------------------------------------------------------


%%%---- 正文行距设置 ------------------------------------%
\baselineskip=20pt  % 正文行距为 20 磅
% %------------------------------------------------------------------------------





%%======中文摘要===========================%
\begin{cnabstract}

你的摘要

\end{cnabstract}
\par
\vspace*{2em}


%%%%--  关键词 -----------------------------------------%%%%%%%%
%%%%-- 注意: 每个关键词之间用“;”分开,最后一个关键词不打标点符号
\cnkeywords{摘要; 摘要; 摘要; 摘要 }


%%====英文摘要==========================%


% \begin{enabstract}
% This thesis is a study on the theory of \dots.

% \end{enabstract}
% \par
% \vspace*{2em}

% %%%%%-- Key words --------------------------------------%%%%%%%
% %%%%-- 注意: 每个关键词之间用“;”分开,最后一个关键词不打标点符号
%  \enkeywords{\LaTeX{};  }
    % 加入摘要, 申明.
%==========================把目录加入到书签==============================%%%%%%
\pdfbookmark[0]{目录}{toc}
\tableofcontents
\mainmatter %% 以下是正文
%%%%%%%%%%%%%%%%%%%%%%%%%%%--------main matter-------%%%%%%%%%%%%

% 引言章
\chapter{引言}

你的引言\upcite{Pawlak1982Rough},\cite{Pawlak1982Rough}

% 模板说明
% !Mode:: "TeX:UTF-8"
%%%%%%%%-------杂七杂八的话--------%%%%%%%%


\chapter{杂七杂八的话}

\section{Readme}

模板文件的结构, 如下表所示:
\begin{table}[ht]\centering
  \begin{tabular}{r|r|l}
    \hline\hline
    \multicolumn{2}{l|}{Bachelor-template.tex } & 主文档. 在其中填写正文.                                      \\ \hline
                                                & frontmatter.tex                      & 郑重声明、中英文摘要. \\ \cline{2-3}
    \raisebox{1em}{includefile 文件夹}          & backmatter.tex                       & 致谢.                 \\ \hline
    \multicolumn{2}{l|}{figures 文件夹}         & 存放图片文件.                                                \\ \hline
    \multicolumn{2}{l|}{WHUBachelor.cls }       & 定义文档格式的 class file. 不可删除.                         \\ \hline\hline
  \end{tabular}
\end{table}

无需也不要改变、移动上述文档的位置.

如果不习惯用~\verb|\include{ }|~的方式加入``子文档'', 当然可以把它们合并在主文档, 成为一个文档.
({\kaishu 但是这样并不会给我们带来方便.})

利用~WinEdt~的~Project tree, 可以方便地管理这些文件:
\begin{itemize}
  \item 点击~WinEdt~窗口的~Project Tree~按键;
  \item 再点击~WinEdt~窗口的~Set Main File~按键;
\end{itemize}
接下来的管理, 已经清楚地展示在跳出的窗口中了. 再去处理其他的文件时, 还要点击~WinEdt~窗口的~Remove Main File~按键.


\section{更新记录}
2016 年 06 月更新: 正文字体为小四号; 英文字体为 Times New Roman; 修订图表标题的字体、字号; 修订目录的字号; 修订附录章节编号的问题.
非常感谢武汉大学数学与统计学院 2012 级张仕俊、林颖倩、宋俍辰等同学.

2016 年 05 月更新: 参考文献加到目录. 感谢武汉大学经济与管理学院的郑中天同学. [上次修订使用的版本有误, 非常抱歉.]

2016 年 02 月更新: 调整为适应 TeX Live 2015 的版本.

2014 年 06 月更新: 修改章节标题、声明标题、图表标题的字体和大小. 再次感谢孙启航同学.

2014 年 05 月更新: 参考文献加到目录. 感谢武汉大学计算机学院孙启航同学、数学与统计学院李振坤同学指出这个纰漏.

2013 年 12 月更新: 加上英文封面. 教务部的写作规范中的附例, 并没有英文封面. 但是遇到很多同学说要加上.


\section{字体调节}

\begin{tabular}{ll}
  \verb|\songti| & {\songti 宋体}   \\
  \verb|\heiti| & {\heiti 黑体}    \\
  \verb|\fangsong| & {\fangsong 仿宋} \\
  \verb|\kaishu| & {\kaishu 楷书}
\end{tabular}


\section{字号调节}
字号命令: \verb|\zihao| \index{zihao}

\begin{tabular}{ll}
  \verb|\zihao{0}|  & \zihao{0}  初号字 English \\
  \verb|\zihao{-0}|  & \zihao{-0} 小初号 English \\
  \verb|\zihao{1} |  & \zihao{1}  一号字 English \\
  \verb|\zihao{-1}| & \zihao{-1} 小一号 English \\
  \verb|\zihao{2} | & \zihao{2}  二号字 English \\
  \verb|\zihao{-2}| & \zihao{-2} 小二号 English \\
  \verb|\zihao{3} | & \zihao{3}  三号字 English \\
  \verb|\zihao{-3}| & \zihao{-3} 小三号 English \\
  \verb|\zihao{4} | & \zihao{4}  四号字 English \\
  \verb|\zihao{-4}| & \zihao{-4} 小四号 English \\
  \verb|\zihao{5} | & \zihao{5}  五号字 English \\
  \verb|\zihao{-5}| & \zihao{-5} 小五号 English \\
  \verb|\zihao{6} | & \zihao{6}  六号字 English \\
  \verb|\zihao{-6}| & \zihao{-6} 小六号 English \\
  \verb|\zihao{7} | & \zihao{7}  七号字 English \\
  \verb|\zihao{8} | & \zihao{8}  八号字 English \\
\end{tabular}

\section{已加入的常用宏包}

\begin{description}
  %  \item[amsmath,amssymb]
  \item[cite]  参考文献引用, 得到形如 [3-7] 的样式.
  \item[color,xcolor]  支持彩色.
  \item[enumerate]  方便自由选择 enumerate 环境的编号方式. 比如

        \verb|\begin{enumerate}[(a)]| 得到形如 (a), (b), (c) 的编号.


        \verb|\begin{enumerate}[i)]| 得到形如 i), ii), iii) 的编号.

\end{description}

另外要说明的是,  itemize, enumerate, description 这三种 list 环境, 已经调节了其间距和缩进,
以符合中文书写的习惯.

\section{标点符号的问题}

建议使用半角的标点符号, 后边再键入一个空格. 特别是在英文书写中要注意此问题!

双引号是由两个左单引号、两个右单引号构成的: \verb|``  ''|. 左单引号在键盘上数字~1 的左边.

但是, 无论您偏向于全角或半角, 强烈建议您使用实心的句号, 只要您书写的是自然科学的文章.
原因可能是因为, 比如使用全角句号的句子结尾处的``$x$。''容易误为数学式~$x_0$(\verb|$x_0$|)吧.



% 总结章
% !Mode:: "TeX:UTF-8"
%%%%%%%%-------结论与展望--------%%%%%%%%



\chapter{结论与展望}

你的总结


%%%============================================================================================================%%%

%%%=== 参考文献 ========%%%
\cleardoublepage\phantomsection

% 加入参考文献页
\addcontentsline{toc}{chapter}{参考文献}
\bibliography{ref.bib}

% 致谢,小文章一般不需要
% \include{includefile/backmatter} %%%致谢

%%%-------------- 附录. 不需要可以删除.-----------
\appendix

\chapter{测试}

\section{第一个测试}
测试公式编号
\begin{equation}
  1+1=2.
\end{equation}

% 表格编号测试

\begin{table}[h]
  \centering
  \caption{测试表格}
  \begin{tabular}{*{20}c}
    \hline
    % after \\: \hline or \cline{col1-col2} \cline{col3-col4} ...
    11 & 13 & 13 & 13 & 13 \\
    12 & 14 & 13 & 13 & 13 \\
    \hline
  \end{tabular}
\end{table}


\chapter{附录测试}

% 测试

\chapter{附录测试}

% 测试

\cleardoublepage
\end{document}



